\documentclass[12pt]{article}

\usepackage[utf8]{inputenc}
\usepackage[english]{babel}
\usepackage{titling}
\usepackage{fancyhdr}

\renewcommand{\baselinestretch}{1.25}

\usepackage[left=0.8in,
right=1.8in,
top=1.in,
bottom=1.in]{geometry}
\setlength{\footskip}{.5in}

\title{The Saint}
\author{Nikolai Usyk}
\date{}

\pagestyle{fancy}
\fancyhf{}
\chead{The Saint}
\fancyfoot[C]{\thepage}

\setlength{\droptitle}{-30pt}

\renewcommand{\familydefault}{\sfdefault}

\pretitle{\begin{center}\fontsize{18bp}{18bp}\selectfont}
	\posttitle{\par\end{center}}

\preauthor{\begin{center}\fontsize{14bp}{14bp}\selectfont}
	\postauthor{\par\end{center}\vspace{-50bp}}

\begin{document}
	
\maketitle

Every year since Sean was born he had attended a funeral. The impressive odor of fresh flowers and dewy broken ground were as familiar to him as the squeaky chime of a swing set. He had seen eight related bodies, uncles, aunts, grand and god parents, laid out pale in their pillowed boxes. He didn’t remember them all though; the first followed quick on his christening. His parents had taken him to the sad ceremony in swaddling clothes praying he wouldn’t shriek. His knowledge of mortality grew with his limbs. On his fourth birthday in the same long carpeted hall, he asked why his grandmother wouldn’t wake up. They told him she was just very tired. But as each successive relative was laid in the box, never to be seen again the truth became evident. By the time Sean was seven he understood that even his fresh skin would one day stiffen  and  be buried. He was horrified  when  he fully grasped  the consequence that one day he would cease to exist when he had only begun to exist. He couldn’t understand not being. It made life seem like a big awful joke. Then he grasped the afterlife which his parents and teachers were always impressing upon him especially at funerals. Suddenly death wasn’t the end, and his grandmother did wake up, just somewhere else. The good go off to live with God in Heaven where everything is the best that it could possibly be. This made life seem less like a joke and more like a test, a test that he was determined to pass.

Sean was given to obsessions. He was driven almost involuntarily to accumulate facts about whatever caught his interest at a given moment. Even if he didn’t fully understand what he knew, he still memorized all the information, names, places, events, about whatever or whoever he was focused on so he could recite them at any moment. A geography lesson caught his attention one day. He liked imagining what life was like in Boise, Idaho or Austin, Texas. Three days later he knew the capital of every state in America by rote, and a month later he had memorized nearly all the capitals of the world and their respective countries although he would have been unable to find many of the countries on the map let alone pronounce the many of names correctly. When he was nine he decided to become the greatest basketball player of all time after watching Michael Jordan play in the NBA Finals. At every opportunity that summer he was outside shooting around at the basketball hoop over the garage in his driveway setting goals and deciding which was a better indicator of future success, making five free throws in a row or making seven free throws out of ten. When his mother called him in for dinner he had to make five free throws in a row before going inside. His mother glared at him from the doorstep and threatened to take away the hoop, but Sean was undeterred. He was determined to be the greatest, but his training was deterred by a fall off his rarely ridden bike which broke his arm.  He was unable to play for two months and quickly forgot about his hoop dream as he sat inside reading a book of vignettes about the presidents.

When Sean was ten he was taken to another funeral. His younger sister, Caitlin, was hit by a careless car outside their home. She was only seven. Sean didn’t cry; he had begun to see death as simply an end to the test of life, the moment the teacher calls “time’s up”. But it wasn’t that he wasn’t sad about his sister’s death, it was that he was stunned and shamefully ambivalent. He passed her room shortly afterward and felt almost happy. The test was over for her, what a relief. But you weren’t supposed to be happy, nobody else looked happy. He was surprised no one else realized it. Sometimes though, her death terrified him. Before, all the funerals he had attended were for old people, but this time she was even younger than him. She barely had time to pick up her pencil and start filling in the circles before God yelled “time’s up.” It was as if God was telling him that his time could be at any moment, that he ought to pick up his pencil and start getting those answers right quick or else he’d be taken unprepared; only one wrong mark could fail him to hell. Because hell is very real and very horrible, a priest had told him. He was sunk in thought, disbelief, and anxiety when the little coffin was laid in the wet broken ground that smelled even more potent than it had before.

The wake followed and the mourning party moved on to their house, where cold cuts and sandwich bread were laid out on the kitchen table. It was crowded as he sat on the couch, his eyes focused on a spot on the far wall. He didn’t respond to anyone who tried to talk to him.  Figures passed in front of him, his former teachers with running mascara, Caitlin’s little friends and classmates who hadn’t yet stopped crying. It was their first funeral, and they couldn’t understand it. He understood that her test was being graded currently, and she would be placed according to her score, but he couldn’t tell anyone that because he understood funerals. The thought spun in his brain into a kind of noise: not only old people died; young ones died too. Now his grandmother, his only surviving grandparent, stood in front of him trying to gain his attention. His mother prodded him on the shoulder.

“Your grandmother is talking to you Sean.”

He snapped out of his reverie and looked at her.  His grandmother spoke, “Sean, honey, I got you this.”  She took out a necklace with a small medallion on the end of it.  “It’s a picture of St. Francis of Assisi there, just like who your school is named after.  He was a great Saint.  You’ve had a hard time; if you pray to him he’ll help you. Ok.  I have to go now, but you be good ok, and pray for your poor sister.”

He stared at her numbly for a second; his mother prodded him again, more forcefully. “Say thank you to your grandmother, Sean. She gave you something.”

“Thank you.”  He said weakly and stared at the tonsured man on the small metal oval.

~

~

The outpouring of sympathy at school was more than he was accustomed to. Teachers couldn’t look him in the face without tearing up. 
His classmates grew quiet around him. 
It was lonely, but he didn’t mind too much. 
It gave him time to think and he had a lot to think about. 
They dedicated all the daily prayers of the school, and all first Friday masses of the year to Caitlin’s memory. 
He wore the medal that his grandmother gave him, and whenever he had free time he shut his eyes and asked St. Francis to intercede on behalf of his sister.

One day he asked his teacher, Sister Mary Theresa, how people become Saints.

“Only very very holy people become Saints, Sean.”  She answered.  “They live life like Jesus said it should be lived. They show us all the way to live a holy life. And after they die, they perform miracles.”

“How can they perform miracles after they’re dead?”

“They’re so holy that God allows them to come back in spirit and help people on Earth.” 

“How did St. Francis become a Saint?”

“Oh, St. Francis is one of the greatest Saints. Everyone recognizes St. Francis. He gave away all his possessions like Jesus said. And he led other people by making an order for monks. And he was always kind to people. And he offered up everything he did to God. He was a true Servant of Christ. He healed people.  He wrote great poems praising God. And he had the stigmata.”

“What’s the stigmata?”

“It’s something that only very holy people get. 
It’s a miracle where the person gets holes in their hands and feet just like Jesus did on the cross. 
They share in the suffering of Christ. It’s a great honor that only the holiest people get.”

The next day Sean took a book out from the school library on St. Francis. He learned that St. Francis was very rich, but then he gave up everything he had. He even gave up his shoes and walked around barefoot. After he died he healed four people, and he exorcised a demon from a woman’s body. He never did anything really amazing when he was alive. He just gave away everything and found some people who wanted join in while being really nice to everyone and praying a lot. And he’s still remembered and revered 800 years later, and he still exists. Saints could come back and do things after they died that they couldn’t do while they were alive.

If God allowed St. Francis to heal people on Earth after he died and went to Heaven, then God must really like St. Francis. 
And what is life for but to please God; He’s the one who created us. 
We’re all alive only because of Him, and He has the power to make our life better or worse and to make our afterlife better or worse. 
The purpose of life is pleasing God, Sean decided.   
He set out to become a greater Saint than even St. Francis.

At first he prayed often because that’s what the book said St. Francis did. When he wasn’t in class or at the dinner table he was praying, eyes closed hands folded contemplating God and His Glory. Even at recess he would sit cross-legged in the corner of the asphalt yard as his classmates played dribbled and tagged. He recited all the prayers he could remember to himself from the Prayer of St. Francis to the Angelus. He carried a rosary around in his pocket and counted the beads in the lunch line. When he woke up in the morning while his mother wasn’t looking he knelt before the statuette of the Virgin Mary in the corner of the living room in his house and prayed for her to intercede on his sister’s behalf. All day he prayed; it was beginning to get a bit lonely because the only person he really talked to was God, and he was never quite sure if God was talking back. He would hear a voice in his head occasionally filling in the other side of the conversation, but he was unsure about whether this was God, an Angel, or someone else. He was determined to overcome his doubt though, that’s what Saints had to do; they had to give up their lives and devote themselves to something that no one could see. They must have felt silly sometimes like he did. But once they got a sign like the stigmata, they probably stopped feeling silly.

But even after two weeks of constant prayer it didn’t seem like he’d made a dent in the brick wall of God’s favor. He had been trying to perform little miracles to see if God had sent him a sign he hadn’t noticed yet like the ability to heal small cuts on his arm that he would make with a kitchen knife, but when he slipped and cut himself too deep he decided instead to attempt to move things or make dice roll to a certain number. It never worked. Once he rolled three nines in a row with the dice, and thought he was on to something but the next twenty rolls came up on the wrong number. So God hadn’t shared any of His divinity with him. Sean kept asking God what He wanted when he was praying, and God told him what He wanted Sean to do. A lot of it sounded like what his mom wanted him to do. So he set the table and made his bed every day and yet no signs. He was tired of not talking to anyone and having his eyes closed so often and kneeling and besides he had run out of things to talk to God about. He had asked him all the questions that he had always wanted to ask, and he thought he understood the answers. Then one day Sean asked God for the hundredth time, what he had to do to become a Saint. God said that Sean already knew what he had to do.

If he already knew what he had to do, then any idea he had could be the right one. He had no way of knowing which idea was right, since he had so many, so he would have to try them all. Sean had read that St. Francis would fast for days, and that he walked around barefoot.  Whenever he could, Sean began to take his shoes off and walk on the roughest pavement or bare ground. At first the loose stones hurt his feet, but he got used to it. He decided to forego lunch and save the lunch money that his parents gave him each day to put in the poor box at church on Sunday. Surely this ensure his seat in Heaven and make God recognize his piousness. Perhaps if he went without food for long enough God would allow him to perform a minor miracle and after that would follow sure Sainthood and immortality. After a week of skipping lunch, which felt like a month, he began to feel like each bite of dinner or breakfast was pulling him further from God. He pushed around his dinner and fed a good portion to the dog; he told his mother that he had eaten breakfast when in fact he hadn’t. It wasn’t hard; she had become easy to trick after his sister died. After two days of not eating he had no energy, when he closed his eyes to pray his mind shut off and he started nodding off to sleep. He shook when he moved, his stomach felt like it was caving in on itself, and he wanted food more than he had ever wanted anything before except maybe to go to Heaven. He kept thinking of the horror of Hell next to the paradise of Heaven and his little sister’s face with a question mark as a warning. If this is what it took to become a Saint, he was prepared.

He sat in his classroom after lunch and almost involuntarily closed his eyes. 
Apparently he had begun snoring as Sister Mary Theresa was talking.  
She came over to his desk and slammed her ruler.

“What in the Lord’s name are you doing sleeping at one o’clock in the afternoon in my class?
Pay attention or I’ll send you to the principal, and she’ll straighten you out.”

Sean jolted awake, but his eyes were still glazed with sleep and he stared at her blankly, like she was a figment of the dream he was just having.  
He saw himself seeing her.

“Why are you looking at me like that? Why are you sleeping in my class? What in the Lord’s name is wrong with you child? Speak.” He snapped out of his dreamlike state. His classmates’ eyes bored into him, and he was humiliated. His stomach hurt so much.

“I’m sleeping because you have nothing to teach me. And I haven’t eaten in two days.” 
He was angry at himself after he said that, but it just came out of him. It was all supposed to be a secret. 
God doesn’t like braggarts.  
His classmates tittered.

His teacher was horrified, “You haven’t eaten in two days!? Aren’t your parents feeding you?” 

“Yeah.  My parents have been feeding me, but I haven’t been eating it.  I haven’t been eating here either because I’m fasting for God, but you wouldn’t know anything about that. 
You think that a uniform makes you closer to God, but it doesn’t. 
You need to give up something you need, but all you do is wear a uniform and yell at kids all day. 
You don’t know anything about God or Jesus.”

He sat back in his seat arms folded across his chest, head turned like a flinch fighting tears of agitation. She looked at Sean for a few seconds and then said quietly, “I think you had better go to the principal and repeat what you just said.”

He walked shakily down the steel staircase to the principal. 
She wasn’t on his side. 
Nuns always stick together. 
She told him that he needed to eat because he needed fuel to be healthy, especially at his age, and that fasting was only a good way to get closer to God when you’re an adult. He said he might not have that long. 
She told him that there were different ways for children to be close to God. 
He could pray often and respect his superiors because God blesses obedience. 
She said that anger, especially at teachers and people close to God was a grave sin and sure to hurt his relationship with God.

Sean said, “But Jesus got angry at the money changers in the temple.”

“I don’t think that this situation is like that at all.” Sean hesitated and she asked, “What’s your favorite food?”

“Pizza.”

“What do you like on your pizza?” 

“Pepperoni.”

She picked up the phone. “Yes, can I order a small pepperoni pizza for delivery to St. Francis of Assisi school?” Pause. “Yes, that’s fine. Thank you.” She turned to Sean, “You’re going to eat this pizza right here in front of me.”

“No, I won’t eat it.  I can’t eat anything.” 

“Why?”

He started to say something but fell into silence.

She picked up the phone again, “I’m calling your mother.”

They sat there until the pizza arrived, but no amount of cajoling would get him to open his mouth before his mother arrived. She came into the room and began yelling at Sean who wasn’t persuaded until she began crying. She cried, and said that she was worried about him and scared that he would end up like his sister because of some crazy stunt. He ate. It tasted better than anything had tasted in his life, but it felt like it was anchoring him to the ground away from God.

He couldn’t fast anymore, his mother watched him like a hawk at breakfast and dinner, and Sister Mary Theresa was no more lenient at lunch in school. 
He felt like a prisoner. 
They were all against him, against his bid for Sainthood, but he would show them.
He kept praying several hours each night when he was supposed to be asleep.  
He felt like he was hearing God more clearly than he had before. 
He asked God for the stigmata. 
He figured that no one had prayed for the stigmata before because of how horrible the pain must be. 
But Sean didn’t care; he would do anything for sainthood. It was the only way he could be sure he wouldn’t go to Hell. 
He would have to be declared a Saint once he had the stigmata. 
When he received the stigmata they would all be proven wrong, and he would redeem himself from that humiliation and show that he was right to have fasted.

So he prayed no less constantly and focused on the idea of the stigmata, and the middle of his hands began to itch. 
So he scratched and rubbed them until they became dry and cracking. 
He walked around all day imagining the pain of Christ, imagining how that pain would feel if it happened to him. 
He dug at his palms out of sympathy for Christ.

One day, about a week and a half after he ate the pizza in the principal’s office, his palms had been so worn down with his clawing that during recess in the concrete schoolyard a trickle of blood wound its way down each palm to the amazement of Sean. Finally his prayers had been answered. 
He stood on a bench and spread his arms.

“Hey, hey everyone. Look, look at my palms. I have the stigmata. Look at my hands, it’s the stigmata.” 
They stopped and looked at him. 
Some looked confused, some whispered to each other, and some laughed. 
One boy yelled, “Shut up you idiot!” 
More kids laughed.

“I’m a Saint now. If you want to go to Heaven you have to follow me.” They had already turned and restarted their four square games.

Sister Mary Theresa came over and grabbed his hands. “Oh no, look at you, you’ve dug your nails into your hands. Why would you do a thing like that? Oh, but it’s not so bad. Just a scratch really. Why don’t you go up to the nurse’s office and she’ll rub it with alcohol and maybe put a band-aid on it.”

He stared at her dumbfounded. 
But her tone was so soft and reasonable that he had no choice but to obey. 
As he walked in he heard whispers and giggles. 
Saints weren’t supposed to be laughed at, they were supposed to be prayed to, revered.

The nurse put some hydrogen peroxide on his hands and it stung. 
She wrapped his hand with an ace bandage so he wouldn’t be able to get to it. 
His mom picked him up. She didn’t say anything. That silence made him feel terrible.
He didn’t know what to do so he asked God when he prayed that night, but he didn’t hear anything so he stopped.

\end{document}